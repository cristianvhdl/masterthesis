\section{Finaler Prototyp} \label{sec:final_prototype}

Der finale Prototyp soll, wie bereits erwähnt, alle zuvor beschriebenen Verfahren zur Realisierung von Überlagerungen in einer Augmented Reality Szene beinhalten. Also muss zunächst eine einfache AR Szene geschaffen werden, in der eine virtuelle Kamera existiert, die die intrinsischen und extrinsischen Eigenschaften der realen Project Tango Kamera zu jeder Zeit entspricht. Außerdem muss das aktuelle Farbbild der RGB Kamera in der Szene im Hintergrund dargestellt werden. Für diese Aufgaben existieren, wie bereits in Kapitel \ref{sec:theory_project_tango} erwähnt, Schnittstellen, die diese Informationen liefern. 

Um eine reale Überdeckung sinnvoll testen zu können, benötigen wir zudem ein virtuelles Objekt in der Szene. Dieses sollte im Idealfall nicht zu einfach gestaltet sein, damit die Verfahren anhand praxisnaher Gegebenheiten verglichen werden können. Zu diesem Zweck sollen Objekte in die App geladen werden können, die in dem Forschungsbereich der Computergraphik typischerweise eingesetzt werden. Typische Modelle sind zum Beispiel der \enquote{Utah Teapot}, \enquote{Stanford Bunny} oder \enquote{Blenders Suzanne}\footnote{List of common 3D test models - https://goo.gl/MsOtSW (26.02.16)}.  Eines dieser Modelle soll in die Szene geladen werden und es soll die Möglichkeit gegeben sein, dass das Objekt flexibel positioniert werden kann. Um das zu realisieren, wird der beschriebene Raypicking Mechanismus für die Auswahlgeste umgesetzt.

Um die Ergebnisse der realen Überlagerung einfach gegenüberstellen zu können, sollen die beschriebenen Tiefenbild generierenden Verfahren flexibel im Betrieb ausgetauscht werden. Dazu gehört das Rendering der Pointcloud Projektion, die TSDF Rekonstruktion durch Chisel und die Ebenenrekonstruktion. Außerdem soll das Filtern des Tiefenbildes mit Hilfe des Guided Filter optional zu jeder Zeit möglich sein. Hilfreich wäre es zudem, die Einstellungen des Guided Filters flexibel anpassen zu können. Wie diese Verfahren und der flexible Austausch umgesetzt wird, wird in den folgenden Kapiteln näher beschrieben.