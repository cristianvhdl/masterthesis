\section{Technische Problemstellungen}

Auch wenn schon einige Probleme in der Umsetzung der jeweiligen Verfahren in Kapitel \ref{sec:method-implementation} näher beschrieben wurden, werden hier noch Einzelheiten aufgegriffen, die bei der Entwicklung für Projekt Tango zu beachten waren. 

Alle von Project Tango zurückgegebenen Vektoren besitzen ihre eigene Konvention bezüglich der Achsenanordnungen. Gegenüber der Konvention in OpenGL sind die Achsen \(Z\) und \(Y\) vertauscht. Außerdem zeigt die resultierende \(Z\)-Achse in die entgegengesetzte Richtung. Aufgrund dieser unterschiedlichen Konventionen müssen alle Vektoren \(\vec{v}\) von Project Tango mit der Transformationsmatrix \(T_{OGL}^{PT}\) aus Gleichung \ref{eq:transformation} konvertiert werden \citep{Proje15:online}. Nachdem Google im Laufe dieser Arbeit neue Schnittstellen\footnote{Project Tango API: Transformation Support - \url{https://goo.gl/N8dapq} (29.02.16)} zur Verfügung gestellt hat, um diese Transformationen zu abstrahieren, können die \enquote{TransformationSupport} Methoden hierfür genutzt werden.

\begin{equation} \label{eq:transformation}
T_{OGL}^{PT} =\left( \begin{matrix} 1&0&0&0\\0&0&-1&0\\0&1&0&0\\0&0&0&1 \end{matrix} \right)
\end{equation}

Wie bei vielen Echtzeitsystemen ist das Problem der Nebenläufigkeit (engl. Concurrency) auch in der Schnittstelle von Project Tango zu beachten. Die API von Project Tango publiziert die Sensordaten typischerweise asynchron. Außerdem werden die Rekonstruktionsverfahren von den entsprechenden API Publikationen angestoßen. Deshalb muss besonders beim Rendering sichergestellt werden, dass die Daten während dieses Prozesses nicht modifiziert werden. Diese Synchronisation wird durch den Einsatz von \enquote{std::mutex} aus der C/C++ Standard-Bibliothek gewährleistet. 

Bei der Anwendung des Guided Filters von OpenCV ist zu beachten, dass die  Speicherkonventionen von OpenGL und OpenCV, was die X und Y Achse angeht, genau vertauscht sind. Dadurch können zum Umwandeln der Bilder in das jeweilige Framework nicht die direkten Adressen verwendet werden, da sonst das Bild um \(90^{\circ}\) gedreht wird. Da dieser Filterprozess jedoch für den Nutzer völlig transparent durchgeführt wird und das Bild wieder in OpenCV zurück konvertiert wird, kann der Filterprozess einfach um \(90^{\circ}\) gedreht durchgeführt werden.
