
\section{Z-Buffer Algorithmus} \label{sec:z-buffer}

Der Z-Buffer Algorithmus, welcher fast zur selben Zeit von \citet{straber1974schnelle} und \citet{catmull1974subdivision} vorgestellt wurde, ermöglicht in der Computergrafik eine einfache Berechnung der Überdeckungen von gerenderten Objekten auf der Bildebene. Hierzu wird neben der Bildebene noch ein sogenannter \enquote{Z-Buffer} eingeführt, der für jeden gerenderten Pixel auf der Bildebene eine Tiefeninformation festhält. Initial enthält der gesamte Z-Buffer die Tiefeninformation die der Backclipping-Ebene entspricht. Somit lässt sich für jeden Pixel einer bestimmten Position bestimmen, ob dieser einen kleineren Tiefenwert besitzt und somit auf die Bildebene gerendert wird. 

Heutzutage ist dieser leicht zu implementierende Mechanismus aus Performancegründen in nahezu jeder Grafikkarte in Hardware implementiert. Das Problem bei dieser einfachen Umsetzung ist jedoch, dass jedes Objekt der Szene den gesamten Renderingprozess durchlaufen muss, auch wenn es später auf der Bildebene nicht erscheint. Abhilfe für dieses Problem bietet der Hierarchical Z-Buffer von \citet{greene1993hierarchical}, welcher nicht sichtbare Objekte zuvor aussortiert. Für die Anwendung für die Optimierungsverfahren in Kapitel \ref{sec:optimization} ist jedoch nur die Idee des einfachen Z-Buffers relevant.