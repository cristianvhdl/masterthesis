\section{Interaktion mit Tiefeninformationen in Augmented Reality}

Wie in Kapitel \ref{sec:ar-interaction} beschrieben, bedarf es bei der Umsetzung von Augmented Reality Systemen ein anderes Interaktionsparadigma. Auch wenn die Entwicklung der neuen Tablet und Smartphone Geräte durch Touchscreens eine neue Interaktionsform eingeführt haben, ist sie in den meisten Fällen auf einer zweidimensionalen Ebene beschränkt. In der Entwicklung von Virtual Reality oder voll virtuellen Anwendungen und Spielen wird oft für die Auswahlgeste der Raycasting Mechanismus verwendet, um eine zweidimensionale Interaktion im dreidimensionalen Raum zu ermöglichen. Darüber hinaus gibt es verbesserte semantische Interaktionsformen basierend auf einer zweidimensionalen Eingabe, wie von \citet{elmqvist2008semantic} beschrieben.\\

Hier soll aber zunächst eine Raycasting Variante für Augmented Reality Anwendungen umgesetzt werden, die nicht von einem kompletten Modell in Form von Polygonen oder anderen Primitiven der realen Umgebung ausgeht. Diese AR Interaktionsmöglichkeit würde das passende Positionieren von virtuellen Objekten im realen Raum ermöglichen. Voraussetzung für die folgende Umsetzung, ist die entsprechende Kalibrierung und Gleichstellung der extrinsischen und intrinsischen Kameraparametern der virtuellen sowie der realen Kamera. \\

Als Erstes wird ein Strahl erzeugt, der durch die Position der virtuellen Kamera und durch den jeweils ausgewählten Punkt auf der Viewingplane läuft. Die Bestimmung ist hier bereits gegeben durch Google Tangos \enquote{Motion Tracking}. 