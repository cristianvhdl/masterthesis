\chapter{Fazit} \label{sec:conclusion}

\section{Evaluation der Verfahren}

Die implementierten Verfahren haben gezeigt, dass mit dem Ansatz der Tiefenbild Überdeckung von \citet{wloka1995resolving} eine Echtzeit Überdeckung virtueller Objekte auf der mobilen Project Tango Hardware erfolgreich umgesetzt werden kann. Dabei wird im Folgenden auf jedes Verfahren sowie ihrer Vor und Nachteile im Kontext der anderen Verfahren und auf Basis der durchgeführten Tests eingegangen. \\

\subsection*{Pointcloud Projektion}

Die Überlagerung durch die Pointcloud Projektion bietet gegenüber den anderen Verfahren den Vorteil, dass sie zu jeder Zeit eine dynamische und aktualisierte Repräsentation der Tiefe der Szene liefert und somit auch Änderungen in der Szene sofort berücksichtigt. Außerdem ist das Verfahren nicht auf Clustergrößen beschränkt und kann dadurch auch auch komplexe Strukturen abbilden. Zu erkennen ist dies bei der Testszene zwei, bei der die Bilddifferenz zum optimalen Ergebnis am geringsten ist, obwohl die Pflanze eine komplexe Struktur besitzt.\\

Dadurch, dass die Pointcloud von Project Tango Fehler in Form von Ausreißern und einem gewissem Rauschen enthalten kann, spiegeln sich diese Fehler auch in der berechneten Projektion wieder. Das führt dazu, dass zum Beispiel die Kante einer realen Überlagerung durchgehend in Bewegung ist und Ihre Struktur mit jedem neuen Tiefenbild variiert. Ein weiteres Problem dieser Technik ist, dass sie, dadurch dass sie sich nur auf einen Datensatz pro Aufnahme bezieht, die Tiefe nur innerhalb des Messbereichs des Tiefensensors repräsentieren kann. Dadurch können Überlagerungen von realen Objekten innerhalb der ersten 50 Zentimeter und ab vier Metern nicht mehr bestimmt werden.\\

\subsection*{Ebenen Rekonstruktion}

Die Ebenen Rekonstruktion löst die Schwächen der Pointcloud Projektion in dem Sinne, dass sie die Ungenauigkeit der Tiefeninformation als eine Oberflächen Approximation mit Hilfe von RANSAC in Form von Ebenen abbildet. Hierdurch werden Ausreißer ignoriert und auch das Rauschen wird durch eine lineare Regression gemittelt. Zusätzlich ermöglicht das Vorgehen der Ebenen Rekonstruktion eine kontinuierliche Verbesserung, indem alle bereits aufgenommenen Pointclouds in die aktuelle Rekonstruktion einfließen. Auch wenn dieser Rekonstruktionsansatz durch den Octree eine Rekonstruktion in einer groben Struktur, den Clustern des Octrees, durchführt, erhält das Verfahren durch das Ermitteln der konvexen Hülle pro gefundener Ebene einen gewissen Detailgrad, um auch schwierige planare Strukturen abbilden zu können. Die gemessenen Ergebnisse der Szenen eins und zwei spiegeln diese positive Eigenschaft wieder und zeigen, dass diese Art der Rekonstruktion auch komplexe Szenen für dieses Testszenario gut abbilden kann.\\

In einem manuellen dynamischeren Test mit Bewegungen weißt dieses Verfahren jedoch einige Schwächen auf. So sind durch die begrenzte Dichte der Pointcloud Lücken zwischen den Ebenen zu sehen, die zwar von Aufnahme zu Aufnahme kleiner werden aber üblicherweise die Oberfläche nicht komplett schließt. Das führt dazu, dass zum Beispiel große Oberflächen, die ein virtuelles Objekt überlagern, das Objekt vereinzelt nicht aussparen, da keine Tiefe an den Stellen durch Lücken zwischen den Ebenen vorhanden sind. Außerdem ist das Verfahren nur bedingt in der Lage runde Strukturen wie den Sitzball aus Szene eins zu rekonstruieren. Diese Fehler werden besonders dann sichtbar, wenn man sich um diesen Ball dreht und er eine Überlagerung mit den Ecken und Kanten der Ebenen aus der Rekonstruktion auf ein virtuelles Objekt ausübt. Neben der fehlenden Unterstützung für runde Konturen besitzt dieses Verfahren keine Möglichkeit, Messungen zu revidieren wenn reale Objekte in der Szene verändert wurden oder ein Drift Fehler von Project Tango auftritt.

\subsection*{TSDF Rekonstruktion}

Wie zu erwarten liefert Chisel als eine TSDF Implementierung, aufgrund der großen Voxelgröße, nicht die Qualität, die zum Beispiel ein KinectFusion liefern kann. Dafür ist es performant genug, um als eine CPU Implementierung eine Echtzeit Rekonstruktion auf der mobilen Project Tango Hardware zu ermöglichen. Genau wie die Ebenen Rekonstruktion bietet Chisel den Vorteil eine Rekonstruktion pro Tiefenbild anzureichern und stetig zu verbessern. Dadurch können Überlagerungen auch außerhalb des Messbereichs des Tiefensensors ermöglicht werden. Anders als bei der Ebenen Rekonstruktion generiert die TSDF Rekonstruktion stets eine geschlossene Oberfläche. Außerdem können runde Strukturen festgehalten werden, wodurch der in Szene eins stehende Sitzball abgebildet werden kann. \\

Die große Voxelgöße führt jedoch dazu, dass wie in beiden getesteten Szenen zu erkennen, die Strukturen der Rekonstrutkion sehr grob ausfallen und die Differenzergebnisse ohne eine Filterung auf einen hohen Fehler hinweisen. Auch wenn Chisel nicht in der Lage ist so detailierte Kantenabbildungen wie die Ebenen Rekonstruktion zu generieren, besitzt Chisel einen Vorteil: Durch den Space Carving Mechanismus können Rekonstruktionen wieder revidiert werden. Das hilft dabei den Problematiken des Drift Effekts von Project Tango entgegenzuwirken. Außerdem könnte durch eine GPU Implementierung auch eine Echtzeit Rekonstruktion mit deutlich kleinerer Voxelgröße und dadurch höherem Detailgrad realisiert werden.


\subsection*{Guided Filter}




* Guided Filter
	* Sehr langsam
	* Führt zu erstaunlichen Optimierungen
	* Weist jedoch je nach Einstellung auch Gradienten auf glatten Tiefenbildern auf


\section{Ausblick}

* Lenovo deployed
* Bilateral Filter in API während Arbeit erschienen => Guided Filter wäre der Performantere Ansatz
* Light-Field Cameras als Kombination für AR!!


	
	