\subsubsection*{Tiefe aus der Pointcloud Projektion}

Wie im Kapitel \ref{sec:pc-projection} erwähnt müssen die Punkte der Project Tango Pointcloud auf die Bildebene projiziert werden und mit einer entsprechenden Tiefenfarbe und einem Radius auf den Tiefenpuffer gezeichnet werden. Dieser Schritt wurde auch bereits in Prototoypen mit den angegebenen Gleichungen umgesetzt. Jedoch kann man sich hierfür das OpenGL Rendering zu Nutze machen und die Projektion OpenGL übernehmen lassen. Denn OpenGL unterstützt für das Rendering neben Polygonen auch primitiven wie Punkte und Linien.  Zudem lässt sich durch einen entsprechenden Fragmentshader die Größe der Punkte entsprechend anpassen.


