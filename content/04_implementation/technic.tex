\section{Technische Umsetzung und Struktur} \label{eq:technic}

Wie bereits erwähnt, basiert das Project Tango System auf Googles Android Betriebsystem. Dies ermöglicht es Anwendungen mit bestehenden und bewerten Technologien wie OpenGL, Rajawali\footnote{Android OpenGL ES 2.0/3.0 Java Engine - https://github.com/Rajawali/Rajawali} oder der Unity Engine entwickeln zu können. Project Tango bietet hierfür drei verschiedene Schnittstellen, in C/C++, Java und Unity (Mono Framework in C\#), um auf die Sensordaten in verschiedenen Umgebungen zugreifen zu können. Im laufe dieser Arbeit wurden alle Schnittstellen mit verschiedensten prototypischen Entwicklungen getestet.\\

Der finale Prototyp wurde letztendlich in C/C++ entwickelt und basiert auf dem Android NDK\footnote{Android Native Development Kit - http://developer.android.com/tools/sdk/ndk/index.html}. Letztendlich greifen die anderen, höher angesiedelten Schnittstellen auf genau die selbe native Implementierung zurück, um sie in Java und Unity zu Verfügung zu stellen. Außerdem ermöglicht die native Entwicklung, neben Performancevorteilen, den vollen Zugriff auf OpenGL Mechanismen, die von Rajawali gegebenenfalls ausgeschlossen werden.\\

Abbildung \ref{fig:structure} zeigt grob den strukturellen Aufbau der Android Applikation. Der obere Teil der Grafik bezieht sich dabei auf den in Java implementierten Teil, der die Nutzeroberfläche, ihre Interaktion und den Renderingcanvas beinhaltet. Das stellt jedoch nur einen kleinen Teil der Anwendung dar, denn alle Interaktionen und Ereignisse werden über ein Java Native Interface (JNI) zum nativen Teil der Anwendung geleitet, welcher die Ansprache der Schnittstellen, die Prozesslogik und das Rendering selbst beinhaltet. \\

\begin{figure}[h]
  \centering
	\includegraphics[width=1.0\textwidth]{content/images/implementation/uml.png} 
  \caption{Struktureller Aufbau des Prototypen}
  \label{fig:structure}
\end{figure}

Die Hauptklasse \enquote{ARApp} in der Grafik widmet sich in der Anwendung nur der Anreicherung und Weiterleitung von JNI Informationen und der Ansprache der Project Tango Schnittstelle. Kern der Anwendung ist die \enquote{Scene} Klasse, welche die Informationen an das entsprechend aktive Verfahren weiterreicht. So wird zum Beispiel die Pointcloud an das Chisel, Pointcloud oder Plane Drawable weitergereicht, damit Sie ein aktualisiertes Tiefenbild rendern können. Auch das Farbbild der Kamera gelangt über die Scene zum RGB Drawable, denn die Szene ermöglicht durch den Einsatz von OpenCV\footnote{Open Source Computer Vision - http://opencv.org/} den optionalen Filter Prozess durch den Guided FIlter.
